%% Copyright 2006-2015 Xavier Danaux (xdanaux@gmail.com).
%
% This work may be distributed and/or modified under the
% conditions of the LaTeX Project Public License version 1.3c,
% available at http://www.latex-project.org/lppl/.
\documentclass[11pt,a4paper]{moderncv}        % possible options include font size ('10pt', '11pt' and '12pt'), paper size ('a4paper', 'letterpaper', 'a5paper', 'legalpaper', 'executivepaper' and 'landscape') and font family ('sans' and 'roman')

\usepackage[utf8]{inputenc}
\usepackage{eurosym}

% moderncv themes
\moderncvstyle{classic}                             % style options are 'casual' (default), 'classic', 'banking', 'oldstyle' and 'fancy'
\moderncvcolor{grey}                               % color options 'black', 'blue' (default), 'burgundy', 'green', 'grey', 'orange', 'purple' and 'red'
%\renewcommand{\familydefault}{\sfdefault}         % to set the default font; use '\sfdefault' for the default sans serif font, '\rmdefault' for the default roman one, or any tex font name
%\nopagenumbers{}                                  % uncomment to suppress automatic page numbering for CVs longer than one page

\moderncvtheme[blue]{classic}
\usepackage[utf8]{inputenc}
% adjust the page margins
\usepackage[scale=0.8]{geometry}
\AtBeginDocument{\recomputelengths}

% personal data
\firstname{Bastian}
\familyname{Köpcke}
\address{Einsteinstraße 62}{48149 Münster, Germany}
%\phone[fixed]{+49~(0)~251~83-32758}
\email{bastian.koepcke@wwu.de}
%\photo[4cm]{passfoto}

\usepackage[backend=biber,
            style=numeric-verb,
            sorting=none, % keep order as in the bib file ...
            giveninits=true,
            defernumbers,
maxbibnames=50]{biblatex}

\defbibenvironment{bibliography}
  {\list%
     {\printfield{year}\hspace{1em}\printtext[labelnumberwidth]{\printfield{labelprefix}\printfield{labelnumber}}}
     {\setlength{\topsep}{0pt}% layout parameters from moderncvstyleclassic.sty
      \setlength{\labelwidth}{\hintscolumnwidth}%
      \setlength{\labelsep}{\separatorcolumnwidth}%
      \leftmargin\labelwidth%
      \advance\leftmargin\labelsep%
      }%
      \sloppy\clubpenalty4000\widowpenalty4000}
  {\endlist}
  {\item}


\DeclareNameAlias{default}{first-last}

\usepackage{xstring}
\usepackage{xpatch}
\newbibmacro*{name:bold}[2]{%
  \def\do##1{\iffieldequalstr{hash}{##1}{\bfseries\listbreak}{}}%
  \dolistloop{\boldnames}%
}

\newcommand*{\boldnames}{}

\xpretobibmacro{name:family}{\begingroup\usebibmacro{name:bold}{#1}{#2}}{}{}
\xpretobibmacro{name:given-family}{\begingroup\usebibmacro{name:bold}{#1}{#2}}{}{}
\xpretobibmacro{name:family-given}{\begingroup\usebibmacro{name:bold}{#1}{#2}}{}{}
\xpretobibmacro{name:delim}{\begingroup\normalfont}{}{}

\xapptobibmacro{name:family}{\endgroup}{}{}
\xapptobibmacro{name:given-family}{\endgroup}{}{}
\xapptobibmacro{name:family-given}{\endgroup}{}{}
\xapptobibmacro{name:delim}{\endgroup}{}{}

% Got hashes from the bbl file
\renewcommand*{\boldnames}{}
\forcsvlist{\listadd\boldnames}
  {{8666bc4e76034dfb97e0acaa5f625527}}

% Only print a year once
\newcounter{currentYear}
\DeclareFieldFormat{year}{%
\ifthenelse{\equal{#1}{\arabic{currentYear}}}%
    {}
%{\setcounter{currentYear}{#1}{\bfseries #1}}}
{\setcounter{currentYear}{#1}{#1}}}

\bibliography{publications}

%----------------------------------------------------------------------------------
%            content
%----------------------------------------------------------------------------------
\begin{document}
\nocite{*}
%-----       resume       ---------------------------------------------------------
\makecvtitle

%----------------------------------------------------------------------------------
%            university education
%----------------------------------------------------------------------------------
\section{University Education}
    \cventry{since 2018}
            {Ph.D. studies}{University of Münster}{Münster, Germany}{}
            {Supervisor: Prof. Sergei Gorlatch, Dr. Michel Steuwer (University of Glasgow)\\
             Main research interests:
             High-level functional performance-portable programming abstractions for high-performance computing.
             I am especially interested in developing a functional pattern-based intermediate array language and formalizing its translation process to imperative code.
             My goal is to have a pattern-based intermediate array language where legal expressions are translated to provably valid GPU programs which achieve high-performance.} 

    \cventry{2014 -- 2018}
		    {Master of Science in computer science}
            {University of Münster}{Münster, Germany}
            {}
%            {\textit{Final grade in computer science: 1.1 (with distinction)}}
            {Thesis title: \textit{Implementing and Optimizing Fast Fourier Transforms in Lift},\\
           In this thesis, I methodically derive expressions for FFTs based on high-level functional primitives and extend the Lift compiler framework to generate high-performance GPU code from the derived expressions.}
%            \textit{Grade for thesis: 1.0}}

	\cventry{2011 -- 2014}
            {Bachelor of Science in computer science}
            {University of Münster}{Münster, Germany}
            {}
%            {\textit{Final grade in computer science:  }}
            {Thesis title: \textit{Implementing SDN-based Multicast in RTF},\\
          In this thesis, I extend the communication layer of the Real-Time Framework, developed at the University of Münster, to support multicast in software defined networks.}
%            \textit{Grade for thesis: }}

%----------------------------------------------------------------------------------
%            publications
%----------------------------------------------------------------------------------
\printbibheading[title={Publications}]
\printbibliography[heading=none]
% Publications from a BibTeX file without multibib
%  for numerical labels: \renewcommand{\bibliographyitemlabel}{\@biblabel{\arabic{enumiv}}}% CONSIDER MERGING WITH PREAMBLE PART
%  to redefine the heading string ("Publications"): \renewcommand{\refname}{Articles}
%\bibliographystyle{plain}
%\bibliography{publications}

\newpage
%----------------------------------------------------------------------------------
%            research visits
%----------------------------------------------------------------------------------
\section{Research Visits}
%    \cventry{from\\-- to}
%			 {Visit Title (amount of time)}
%		     {Institution}
%		     {location}{}
%			 {Short Description}
    \cventry{08/2019\\-- 10/2019}
			{Visiting Researcher (5 weeks)}
		    {University of Glasgow}
		    {Glasgow, UK}{}
			{Funded by HiPEAC\\
            During this visit, I investigated the translation of a functional pattern-based intermediate representation for the Lift language into an intermediate imperative language from which OpenCL code is generated.
            My main focus was on higher control over memory allocations (e.g. OpenCL address spaces) in the functional language to generate high-performance OpenCL kernels for GEMM.}
    \cventry{01/2019\\-- 03/2019}
			{Visiting Researcher (2 months)}
		    {University of Glasgow}
		    {Glasgow, UK}{}
			{Funded by HPC-Europa3\\
             During this visit, I started extending a newly designed compiler and intermediate representation for the Lift language towards a working OpenCL code generation.
             My main focus was on making previously hidden compiler decisions, such as parallelism and copies between memory regions, explicit on the functional language level.}


%----------------------------------------------------------------------------------
%            presentations
%----------------------------------------------------------------------------------
\section{Presentations}
%    \cventry{MM/YYYY}
%            {what}{}{}{}
%            {where}
    \cventry{04/2019}
            {\normalfont{Talk: \textit{Generating Efficient FFT GPU Code with Lift}}}{}{}{}
            {International Workshop on Functional High-Performance and Numerical Computing (FHPNC), Berlin, Germany}
    \cventry{07/2019}
            {\normalfont{Talk: \textit{Generating Efficient FFT Code for GPU from High-Level, Pattern-Based Abstractions}}}{}{}{}
            {International Symposium on High-Level Parallel Programming and Applications (HLPP), Linköping, Sweden}
    \cventry{02/2019}
            {\normalfont{Invited Talk: \textit{Programming Specialized Hardware Using Lift}}}{}{}{}
            {Edinburgh Parallel Computing Centre (EPCC), Edinburgh, UK}

%----------------------------------------------------------------------------------
%            research projects
%----------------------------------------------------------------------------------
\section{Research Projects}
\cventry{since 03/2018}{LIFT}{\textit{A Novel Approach to Achieving Performance Portability on Accelerators}}{}{}{
	Ongoing research, \textit{www.lift-project.org}\\
	The Lift project is a novel approach to generate high-performance OpenCL kernels from high-level functional programs.
  As one of the main developers of the project, I have been mostly working on extending Lift in order to express complex high-performance applications with a particular focus on Fast Fourier Transforms.
	}

\cventry{2014 -- 2017}{KETTI}{\textit{Competence Development of Student Teaching Assistants in Computer Science}}{}{}{
  Project webpage, \textit{www.uni-muenster.de/Ketti/en/index.html}\\
  As a student and later research assistant, I assisted in qualitative and quantitative research towards the implementation of a competence model for teaching assistants.
The aim of KETTI is to formalize the preparation of teaching assistants to activate students and support their learning.
}

%----------------------------------------------------------------------------------
%            academic events
%----------------------------------------------------------------------------------
\section{Attended Academic Events And Scientific Committees}
    \cvitem{2019}{Artifact Evaluation Committee -- CGO 2020}
    \cvitem{}{FHPNC -- \textit{International Workshop on Functional High-Performance and Numerical Computing}, Berlin, Germany}
    \cvitem{}{ICFP -- \textit{International Conference on Functional Programming}, Berlin, Germany}
    \cvitem{}{SPLV -- \textit{Scottish Summer School on Programming Languages and Verification}, Glasgow, UK}
    \cvitem{}{ACACES Summer School (organized by HiPEAC) -- \textit{Fifteenth International Summer School on Advanced Computer Architecture and Compilation for High-Performance and Embedded Systems}, Fiuggi, Italy}
    \cvitem{}{HLPP -- \textit{International Symposium on High-Level Parallel Programming and Applications}, Linköping, Sweden}
    \cvitem{}{CSW (organized by HiPEAC) -- \textit{Computing Systems Week}, Edinburgh, UK}
    \cvitem{}{SPLS -- \textit{Scottish Programming Languages Seminar}, St. Andrews, UK}
    \cvitem{2015}{PRACE course -- \textit{Advanced Parallel Programming with MPI and OpenMP}, Jülich Supercomputing Centre, Germany}
    \cvitem{}{BTW -- \textit{16th Conference on Database Systems for Business, Technology, and Web}, Hamburg, Germany}

%----------------------------------------------------------------------------------
%            teaching
%----------------------------------------------------------------------------------
\section{Teaching}
  \cvitem{Winter '20}{Supervised a student project: \textit{Developing a CUDA Backend for Lift Targeting TensorCores}}
  \cvitem{Winter '20}{Teaching assistant for the course: \textit{Introduction to Programming with Java}}
  \cvitem{Summer '19}{Teaching assistant for the course: \textit{Distributed Systems}}
  \cvitem{Summer '19}{Teaching assistant for the course: \textit{Introduction to Programming with C and C++}}
  \cvitem{Winter '19}{Teaching assistant for the course: \textit{Introduction to Programming with Java and Haskell}}
  \cvitem{Winter '15}{Student assistant for the course: \textit{Introduction to Programming}}
  \cvitem{Summer '14}{Student assistant for the course: \textit{Data Structures and Algorithms}}
\end{document}
