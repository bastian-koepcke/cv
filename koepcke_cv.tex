%% Copyright 2006-2015 Xavier Danaux (xdanaux@gmail.com).
%
% This work may be distributed and/or modified under the
% conditions of the LaTeX Project Public License version 1.3c,
% available at http://www.latex-project.org/lppl/.
\documentclass[11pt,a4paper,sans]{moderncv}        % possible options include font size ('10pt', '11pt' and '12pt'), paper size ('a4paper', 'letterpaper', 'a5paper', 'legalpaper', 'executivepaper' and 'landscape') and font family ('sans' and 'roman')

\usepackage[utf8]{inputenc}

% moderncv themes
\moderncvstyle{classic}                             % style options are 'casual' (default), 'classic', 'banking', 'oldstyle' and 'fancy'
\moderncvcolor{grey}                               % color options 'black', 'blue' (default), 'burgundy', 'green', 'grey', 'orange', 'purple' and 'red'
%\renewcommand{\familydefault}{\sfdefault}         % to set the default font; use '\sfdefault' for the default sans serif font, '\rmdefault' for the default roman one, or any tex font name
%\nopagenumbers{}                                  % uncomment to suppress automatic page numbering for CVs longer than one page

\moderncvtheme[blue]{classic}
\usepackage[utf8]{inputenc}
% adjust the page margins
\usepackage[scale=0.8]{geometry}
\AtBeginDocument{\recomputelengths}

% personal data
\firstname{Bastian}
\familyname{Köpcke}
\address{Einsteinstraße 62}{48149 Münster, Germany}
\phone[fixed]{+49~(0)~251~83-32758}
\email{bastian.koepcke@wwu.de}
%\photo[4cm]{passfoto}

\usepackage[backend=biber,
            style=numeric-verb,
            sorting=none, % keep order as in the bib file ...
            giveninits=true,
            defernumbers,
maxbibnames=50]{biblatex}

\defbibenvironment{bibliography}
  {\list%
     {\printfield{year}\hspace{1em}\printtext[labelnumberwidth]{\printfield{labelprefix}\printfield{labelnumber}}}
     {\setlength{\topsep}{0pt}% layout parameters from moderncvstyleclassic.sty
      \setlength{\labelwidth}{\hintscolumnwidth}%
      \setlength{\labelsep}{\separatorcolumnwidth}%
      \leftmargin\labelwidth%
      \advance\leftmargin\labelsep%
      }%
      \sloppy\clubpenalty4000\widowpenalty4000}
  {\endlist}
  {\item}


\DeclareNameAlias{default}{first-last}

\usepackage{xstring}
\usepackage{xpatch}
\newbibmacro*{name:bold}[2]{%
  \def\do##1{\iffieldequalstr{hash}{##1}{\bfseries\listbreak}{}}%
  \dolistloop{\boldnames}%
}

\newcommand*{\boldnames}{}

\xpretobibmacro{name:family}{\begingroup\usebibmacro{name:bold}{#1}{#2}}{}{}
\xpretobibmacro{name:given-family}{\begingroup\usebibmacro{name:bold}{#1}{#2}}{}{}
\xpretobibmacro{name:family-given}{\begingroup\usebibmacro{name:bold}{#1}{#2}}{}{}
\xpretobibmacro{name:delim}{\begingroup\normalfont}{}{}

\xapptobibmacro{name:family}{\endgroup}{}{}
\xapptobibmacro{name:given-family}{\endgroup}{}{}
\xapptobibmacro{name:family-given}{\endgroup}{}{}
\xapptobibmacro{name:delim}{\endgroup}{}{}

% Got hashes from the bbl file
\renewcommand*{\boldnames}{}
\forcsvlist{\listadd\boldnames}
  {{dfe6c38b200b67c8cafef47f5ff406e3}}

% Only print a year once
\newcounter{currentYear}
\DeclareFieldFormat{year}{%
\ifthenelse{\equal{#1}{\arabic{currentYear}}}%
    {}
%{\setcounter{currentYear}{#1}{\bfseries #1}}}
{\setcounter{currentYear}{#1}{#1}}}

\bibliography{publications}

%----------------------------------------------------------------------------------
%            content
%----------------------------------------------------------------------------------
\begin{document}
\nocite{*}
%-----       resume       ---------------------------------------------------------
\makecvtitle

%----------------------------------------------------------------------------------
%            university education
%----------------------------------------------------------------------------------
\section{University Education}
  \cventry{since 2018}
          {Ph.D. studies}{University of Münster}{Münster, Germany}{}{
          Supervisor: Prof. Sergei Gorlatch\\
          Main research interests: High-level performance-portable programming abstractions for high-performance computing,
          Programming and optimization of programs for modern multi- and many-core processors.}
	\cventry{2014 -- 2018}
					{Master of Science in computer science}
					{University of Münster}{Münster, Germany}
					{\textit{Final grade in computer science: --tbd--}}
					{Thesis title: \textit{Implementing and Optimizing Fast Fourier Transforms in Lift},\\
           In this thesis, I methodically derive expressions for FFTs based on high-level functional primitives and extend the Lift compiler framework to generate high-performance GPU code from the derived expressions. \textit{Grade for thesis: --tbd--}}

	\cventry{2011 -- 2014}
					{Bachelor of Science in computer science}
					{University of Münster}{Münster, Deutschland}
					{\textit{Final Grade in computer science: gut(2,0)}}
					{Thesis title: \textit{Implementing SDN-based Multicast in RTF},\\
          In this thesis, I extend the communication layer of the Real-Time Framework, developed at the University of Münster, to support multicast in software defined networks. \textit{Grade for thesis: gut(1,7)}
					}
%----------------------------------------------------------------------------------
%            publications
%----------------------------------------------------------------------------------
\printbibheading[title={Publications}]
\printbibliography[heading=none]
% Publications from a BibTeX file without multibib
%  for numerical labels: \renewcommand{\bibliographyitemlabel}{\@biblabel{\arabic{enumiv}}}% CONSIDER MERGING WITH PREAMBLE PART
%  to redefine the heading string ("Publications"): \renewcommand{\refname}{Articles}
%\bibliographystyle{plain}
%\bibliography{publications}

%----------------------------------------------------------------------------------
%            research projects
%----------------------------------------------------------------------------------
\section{Research Projects}
\cventry{since 03/2018}{LIFT}{\textit{A Novel Approach to Achieving Performance Portability on Accelerators}}{}{}{
	Ongoing research, \textit{www.lift-project.org}\\
	The Lift project is a novel approach to generate high-performance OpenCL kernels from high-level functional programs.
	My contribution to the project has been focused on using Lift to express complex high-performance applications at the example of Fast Fourier Transforms.
	}

\cventry{2014 -- 2017}{KETTI}{\textit{Competence Development of Student Teaching Assistants in Computer Science}}{}{}{
  Project webpage, \textit{https://www.uni-muenster.de/Ketti/en/index.html}\\
  As a student and later research assistant, I assisted in qualitative and quantitative research towards the implementation of a competence model for teaching assistants.
  The aim of KETTI is to formalize the preparation of teaching assistants to activate students and support their learning.
}

%----------------------------------------------------------------------------------
%            teaching
%----------------------------------------------------------------------------------
\section{Teaching}
  \cvitem{Winter 2018}{Teaching assistant for the course: \textit{Introduction to Java}}
	\cvitem{Winter 2014}{Student teaching assistant for the course: \textit{Introduction to Programming}}
  \cvitem{Summer 2014}{Student teaching assistant for the course: \textit{Data Structures and Algorithms}}

%----------------------------------------------------------------------------------
%            academic events
%----------------------------------------------------------------------------------
\section{Attended Academic Events}
\cvitem{12/2015}{PRACE course -- \textit{Advanced Parallel Programming with MPI and OpenMP}, Jülich Supercomputing Centre, Germany}

\cvitem{03/2015}{BTW -- \textit{16th Conference on Database Systems for Business, Technology, and Web}, Hamburg, Germany}

%----------------------------------------------------------------------------------
%            technical skills
%----------------------------------------------------------------------------------
\section{Technical Skills}
	\cventry{Programming Languages}
					{Scala, C/C++, Java, Python}{}{}{}
					{Experiences:
					 Fast Fourier Transforms in Lift (Scala),
           Python C-API used for the simulation of interface accesses in the RTF Multicast Module,
					 Profiling library for OpenCL programs (C++),
				 	 SDN-based Multicast Modul for the Real-Time Framework (C/C++),
					 Implementation of the WiPo architecture (Java),
					}

	\cventry{Parallel Programming}
					{OpenCL, CUDA, OpenMP, MPI}{}{}{}
					{Experience:
					 Performance portability evaluation of OpenCL Kernels on NVIDIA GPUs. JIT compilation of a DSL using LLVM and CUDA Driver API.
					}

\cvline{}{}{}
\cvline{}{}{}
\cvline{}{}{}
 
\end{document}
